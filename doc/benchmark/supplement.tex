\documentclass[11pt]{article}
\usepackage[margin=2cm,a4paper]{geometry}
\usepackage{graphicx}
\usepackage{authblk}
\usepackage{url}
\usepackage{array}
\usepackage{amsfonts}
\usepackage{multirow}
\usepackage{amsmath}
\usepackage[usenames, dvipsnames]{color, colortbl}
\usepackage[normalem]{ulem}

\usepackage[outdir=./img/]{epstopdf}
\usepackage{epsfig}

\usepackage{tikz}
\usetikzlibrary{fit,positioning}
\usetikzlibrary{arrows}

\makeatletter
\def\title#1{\gdef\@title{Supplement to: ``#1''}}
\makeatother
\renewcommand{\thetable}{S\arabic{table}}%
\renewcommand{\thefigure}{S\arabic{figure}}%
\renewcommand{\thepage}{S\arabic{page}}

\setlength\parindent{0pt}

\title{Triqler for Protein Summarization of Data from Data Independent Acquisition Mass Spectrometry}
\author{Patrick Truong \and Matthew The \and Lukas K\"{a}ll}


\begin{document}

\maketitle

\section*{Note S1: xxx}
\label{sec:fc-eval}


\subsection*{Fold change distributions}
\begin{figure}[hbt]
    \centering
    \begin{tabular}{lclc} 
        A & \includegraphics[width=0.4\linewidth]{../../result/report_plots/osw_triqler_intensity.png} & 
        E & \includegraphics[width=0.4\linewidth]{../../result/report_plots/diann_triqler_intensity.png} \\ 
        B & \includegraphics[width=0.4\linewidth]{../../result/report_plots/osw_msqrobsum_intensity.png} & 
        F & \includegraphics[width=0.4\linewidth]{../../result/report_plots/diann_msqrobsum_intensity.png} \\ 
        C & \includegraphics[width=0.4\linewidth]{../../result/report_plots/osw_msstats_intensity.png} & 
        G & \includegraphics[width=0.4\linewidth]{../../result/report_plots/diann_msstats_intensity.png} \\ 
        D & \includegraphics[width=0.4\linewidth]{../../result/report_plots/osw_top3_intensity.png} &
        H & \includegraphics[width=0.4\linewidth]{../../result/report_plots/diann_top3_intensity.png} 
    \end{tabular}
   
    \caption{{\bf Comparison of reported fold change distributions.} We used peptide data from (A-D) DDA spectrum libraries and (E-H) psedo spectra as generated by 
    (A,E) Triqler, (B,F) MSqRobSum, (C,G) MSstats, and (D,H) Top-3. \label{fig:fc_histogram}}
\end{figure}



\subsubsection*{Differential abundance}
\begin{figure}[hbt]
    \centering
    \begin{tabular}{lclc} 
        A & \includegraphics[width=0.4\linewidth]{../../result/report_plots/osw_de_all.png} & 
        E & \includegraphics[width=0.4\linewidth]{../../result/report_plots/diann_de_all.png} \\ 
        B & \includegraphics[width=0.4\linewidth]{../../result/report_plots/osw_de_ecoli.png} & 
        F & \includegraphics[width=0.4\linewidth]{../../result/report_plots/diann_de_ecoli.png} \\ 
        C & \includegraphics[width=0.4\linewidth]{../../result/report_plots/osw_de_yeast.png} & 
        G & \includegraphics[width=0.4\linewidth]{../../result/report_plots/diann_de_yeast.png} \\ 
        D & \includegraphics[width=0.4\linewidth]{../../result/report_plots/osw_de_human.png} &
        H & \includegraphics[width=0.4\linewidth]{../../result/report_plots/diann_de_human.png} 
    \end{tabular}
    \caption{{\bf Comparison of reported differential abundance.} Differential abundance is reported for (A-D) DDA spectrum libraries and (E-H) pseudo spectra for each proteome 
    (A,E) All, (B,F) ECOLI, (C,G) YEAST, and (D,H) HUMAN. \label{fig:da_lineplot}}
\end{figure}

\subsubsection*{Comparison of ability to differentiate differentially abundant proteins}
\begin{figure}[hbt]
    \centering
    \begin{tabular}{lclc} 
        A & \includegraphics[width=0.4\linewidth]{../../result/report_plots/osw_de_human_vs_ecoli.png} & 
        D & \includegraphics[width=0.4\linewidth]{../../result/report_plots/diann_de_human_vs_ecoli.png} \\ 
        B & \includegraphics[width=0.4\linewidth]{../../result/report_plots/osw_de_human_vs_yeast.png} & 
        E & \includegraphics[width=0.4\linewidth]{../../result/report_plots/diann_de_human_vs_yeast.png} \\
        C & \includegraphics[width=0.45\linewidth]{../../result/report_plots/osw_de_human_vs_ecoli_and_yeast.png} & 
        F & \includegraphics[width=0.45\linewidth]{../../result/report_plots/diann_de_human_vs_ecoli_and_yeast.png} \\ 

    \end{tabular}
    \caption{{\bf Comparison of ability to differentiate differentially abundant proteins} We plotted the number of reported differentially abundant  {\em E. Coli} and Yeast proteins as a function of number of proteins from the HeLa background when sorting according to significance for (A) DDA generated spectral libraries and (B) DIA-Umpire geneated Pseudo spectra. For the test we selected a fold-change treshold of 0.4 for Triqler. \label{fig:diff_vs_hela}}
\end{figure}

\subsubsection*{Comparison of statistical calibration}
\begin{figure}[hbt]
    \centering
    \centering
    \begin{tabular}{lclc} 
        %\includegraphics[width=0.3\linewidth]{../../result/report_plots/de_human_vs_de_specie.png} & 
        %\includegraphics[width=0.3\linewidth]{../../result/report_plots/de_human_vs_de_specie.png} \\ 
        %A & B
        A \includegraphics[width=0.5\linewidth]{../../result/report_plots/osw_FP_DE_yeast.png} & &%\includegraphics[width=0.3\linewidth]{} & 
        D \includegraphics[width=0.5\linewidth]{../../result/report_plots/diann_FP_DE_yeast.png} & \\%\includegraphics[width=0.3\linewidth]{} \\ 
        B \includegraphics[width=0.5\linewidth]{../../result/report_plots/osw_FP_DE_ecoli.png} & &%\includegraphics[width=0.3\linewidth]{} & 
        E \includegraphics[width=0.5\linewidth]{../../result/report_plots/diann_FP_DE_ecoli.png} & \\%\includegraphics[width=0.3\linewidth]{} \\ 
        C \includegraphics[width=0.5\linewidth]{../../result/report_plots/osw_FP_DE_all.png} & &%\includegraphics[width=0.3\linewidth]{} & 
        F \includegraphics[width=0.5\linewidth]{../../result/report_plots/diann_FP_DE_all.png} & \\%\includegraphics[width=0.3\linewidth]{} \\ 
    \end{tabular}
  \caption{{\bf Comparison of calibration of the compared summarization methods.} We plotted the fraction of reported differentially abundant HeLa proteins as a function of $q$~value treshhold for (A-C) DDA generated spectral libraries and (D-F) DIA-Umpire geneated Pseudo spectra. \label{fig:frac_hela_vs_fdr}}
\end{figure}

\subsubsection*{Constant variance}
\begin{figure}[hbt]
    \centering
    \centering
    \begin{tabular}{lclc} 
        A \includegraphics[width=0.5\linewidth]{../../result/mu_sigma_variance_plots/osw_log/osw_boxplot_qvalFiltered_pepFiltered_qbinned.png} & &%\includegraphics[width=0.3\linewidth]{} & 
        B \includegraphics[width=0.5\linewidth]{../../result/mu_sigma_variance_plots/osw/osw_boxplot_nolog_qvalFiltered_pepFiltered_qbinned.png} & \\%\includegraphics[width=0.3\linewidth]{} \\ 
        C \includegraphics[width=0.5\linewidth]{../../result/mu_sigma_variance_plots/diann_log/diann_boxplot_qvalFiltered_pepFiltered_qbinned.png} & &%\includegraphics[width=0.3\linewidth]{} & 
        D \includegraphics[width=0.5\linewidth]{../../result/mu_sigma_variance_plots/diann/diann_boxplot_nolog_qvalFiltered_pepFiltered_qbinned.png} & \\%\includegraphics[width=0.3\linewidth]{} \\ 
    \end{tabular}
  \caption{{\bf Uniform offset in standard deviation.} We plotted the standard deviation as a function of the mean of every peptide intensity in the TripleTOF6600 section of the LFQ Bench set  in a linear and log-log scale for (A-B) spectral library and (C-D) pseudo-spectra workflows.  We observe a nearly uniform offset in standard deviation across the intensity scale, demonstrating that $\log(\sigma) \approx \log(\mu) + \log(k)$ and hence   $\sigma \approx \mu k$. Linear scale plots are reported as reference.  \label{fig:mu_sigma_boxplot}}
\end{figure}


\begin{figure}[hbt]
    \centering
    \centering
    \begin{tabular}{lclc} 
        A \includegraphics[width=0.5\linewidth]{../../result/mu_sigma_variance_plots/osw_log/osw_kde_qvalFiltered_pepFiltered_qbinned.png} & &%\includegraphics[width=0.3\linewidth]{} & 
        B \includegraphics[width=0.5\linewidth]{../../result/mu_sigma_variance_plots/osw/osw_kde_nolog_qvalFiltered_pepFiltered_qbinned.png} & \\%\includegraphics[width=0.3\linewidth]{} \\ 
        C \includegraphics[width=0.5\linewidth]{../../result/mu_sigma_variance_plots/diann_log/diann_kde_qvalFiltered_pepFiltered_qbinned.png} & &%\includegraphics[width=0.3\linewidth]{} & 
        D \includegraphics[width=0.5\linewidth]{../../result/mu_sigma_variance_plots/diann/diann_kde_nolog_qvalFiltered_pepFiltered_qbinned.png} & \\%\includegraphics[width=0.3\linewidth]{} \\ 
    \end{tabular}
  \caption{{\bf Gaussian kernel density estimates of quantile binned peptide intensities.} We plotted the Gaussian kernel estimates of the quantile binned peptide intensities for (A-C) spectral library and (C-D) pseudo-spectra workflows. Approximately equal densities in the bins demonstrates the uniform offset in standard deviation across the intensity scale. \label{fig:mu_sigma_KDE}}
\end{figure}

\begin{figure}[hbt]
    \centering
    \centering
    \begin{tabular}{lclc} 
        A \includegraphics[width=0.5\linewidth]{../../result/mu_sigma_variance_plots/osw_log/osw_violinOverlap_qvalFiltered_pepFiltered_qbinned.png} & &%\includegraphics[width=0.3\linewidth]{} & 
        B \includegraphics[width=0.5\linewidth]{../../result/mu_sigma_variance_plots/osw/osw_violinOverlap_nolog_qvalFiltered_pepFiltered_qbinned.png} & \\%\includegraphics[width=0.3\linewidth]{} \\ 
        C \includegraphics[width=0.5\linewidth]{../../result/mu_sigma_variance_plots/diann_log/diann_violinOverlap_qvalFiltered_pepFiltered_qbinned.png} & &%\includegraphics[width=0.3\linewidth]{} & 
        D \includegraphics[width=0.5\linewidth]{../../result/mu_sigma_variance_plots/diann/diann_violinOverlap_nolog_qvalFiltered_pepFiltered_qbinned.png} & \\%\includegraphics[width=0.3\linewidth]{} \\ 
    \end{tabular}
  \caption{{\bf Overlapping violin plots of quantile binned peptide intensities.} We plotted overlapping violin plots of the quantile binned peptide intensities for (A-C) spectral library and (C-D) pseudo-spectra workflows. This despicts the same results as \ref{fig:mu_sigma_boxplot} and \ref{fig:mu_sigma_KDE}. \label{fig:mu_sigma_KDE}} 
\end{figure}





\end{document}
