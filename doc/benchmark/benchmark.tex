
\documentclass[10pt,letterpaper]{article}
\usepackage[top=0.85in,left=2.75in,footskip=0.75in]{geometry}
% amsmath and amssymb packages, useful for mathematical formulas and symbols
\usepackage{amsmath,amssymb}
% Use adjustwidth environment to exceed column width (see example table in text)
\usepackage{changepage}
% Use Unicode characters when possible
\usepackage[utf8x]{inputenc}
% textcomp package and marvosym package for additional characters
\usepackage{textcomp,marvosym}
% cite package, to clean up citations in the main text. Do not remove.
\usepackage{cite}
% Use nameref to cite supporting information files (see Supporting Information section for more info)
\usepackage{nameref,hyperref}
% line numbers
%\usepackage[right]{lineno}



% color can be used to apply background shading to table cells only
\usepackage[table]{xcolor}

% array package and thick rules for tables
\usepackage{array}

\usepackage{dsfont}


%% END MACROS SECTION

\title{Triqler for Data Independent Aquisition Data}
\author{Patrick Truong \and Matthew The \and Lukas K\"{a}ll}


\begin{document}
%\linenumbers
\maketitle

\begin{abstract}
  In this study we show that Triqler, a protein quantification and differential analysis tool based on probabilistical graphical models, has better performance than other protein quantification tools. To show this we compare different processing pipelines using different underlying concept for protein idenfication and quantification... 
\end{abstract}
  

\section*{Introduction}
Label-free quantification (LFQ) using Mass spectrometry (MS) based proteomics has been shown to be an effective methods for studying the relative concentration of proteins in complex mixtures. Compared to Data-dependent axquisition (DDA), Data-independent acquisition (DIA) mass spectrometry allows for a broader dynamical range and more reproducible peptide detection 
´ \citep{zhang2020DIA, Lu2021DIAmeter}.  
 
Triqler is a novel software that uses a probabilitical graphical model for protein quantification and differential expression analysis, essentially eliminating the need for filtering, tresholding and imputational procedures required by many conventional methods. Triqler has been shown to distinguish more proteins for DDA data compared to other DDA protein quantification methods. \citep{The2018Integrated}. 

 
\section*{Materials and methods}
\textbf{Data description}

The data is a DIA dataset used in a previous benchmark of DIA protein quantification benchmarking study [LFQBenchPaper2016]. It is available from the ProteomeXchange Consortium with the dataset identifier PXD002952. The instrumentation used process the data was TTOF6600 system with 32 fixed windows. In the repository the data we use is referred to as the HYE124 hybrid proteome samples. It consists of tryptic peptides with the following ratios: Sample A composed of 65\% w/w, 30\% w/w yeast, and 5\% w/w E. coli proteins. Sample B was composed of 65\% w/w, 15\% w/w yeast and 20\% w/w E. coli proteins. Further details about mass spectrometric instrumentation and data acquisition is available in Navarro et al. [LFQBenchPape2016].     

\textbf{Data preparation and spectral library generation}
The .wiff files are converted to .mzML files in centroided format using msconvert (using windows OS msconver version X.X) with the following options: [check options]. 

Two approaches was used for spectra library generation: DDA acquisition based spectral library generation and Prosit-based spectral library generation using only .fasta file [cite prosit paper]. 

DDA acquisitions of samples from each specie (human, yeast, E. coli) was provided in triplicates for spectral library generation. Uniprot fasta files with one protein seqeunce per gene was concatenated for each specie (UP000005640, UP000000625 and UP000002311, acquired on 2021-06-16).To control for the effect of different protein inference strategies (protein group, parsimony etc.) a modified .fasta file, without shared peptides, was used for database search. The unfiltered fasta files contained 20 590 human proteins, 6 046 yeast proteins and 4 373 E. coli proteins. After filtering the fasta file contained 20 302 proteins (-288 human proteins), 5 848 yeast proteins (198 yeast proteins) and 4 306 E. Coli proteins (-67 E. Coli proteins). Each sequence with length $>$7 amino acids mapping only to one protein. The fasta file contained reverse sequences as decoys for target-decoy analysis. MSFragger with parameters: [check parameters] was used for DDA-search, statistical validation was performed by peptide prophet and protein prophet, and EasyPQP with parameters: [check parameters] was used for spectral library building. OpenSwathDecoyGenerator was used with default setting to generate decoys for the resulting spectral libraries.  

For Prosit-based spectral library generation, the fasta file was converted to prosit input format using encyclopeDIA converter. Prosit_2020_intensity_cid model was used as intensity prediction model and Prosit_2019_irt was used as iRT predicition model.  


\textbf{OpenSwath Analysis}
Version (version) of OpenSwath was used. The spectral library generated above is converted to .pqp format using TargetedFileConverter. Data analysis was conducted using OpenSwathWorkflow with parameters (parameters). After data extraction the data .osw output was merged using pyprophet merge option and pyprophet was used for statistical validation. 

\textbf{DIAUmpire and DIA-NN analysis}
DIAUmpire signal extraction (SE) was used through Fragpipe GUI....
\textbf{EncyclopeDIA and PECAN analysis}

\textbf{Protein quantification}

\textbf{Triqler}

\textbf{Top3}

\textbf{Msstat}

\textbf{Msqrobsum}



\section*{Results}
\textbf{OpenSwath Analysis}

\textbf{DIAUmpire and DIA-NN analysis}

\textbf{EncyclopeDIA and PECAN analysis}


\section*{Discussion}

\section*{Acknowledgements}


\section*{Funding}

This work has been supported by a grant from the Swedish Foundation for Strategic Research (BD15-0043).

\section*{Supporting information}

\bibliographystyle{plain}
\bibliography{transformers}

\end{document}

